\chapter{Podsumowanie}

\section{Wnioski}

Autoenkoder wariacyjny jest bardzo interesującym modelem bazującym na rachunku prawdopodobieństwa, który stara się je oddolnie oszacowywać dla wystąpienia danego zjawiska. Jest to przydatne w problemie znajdywania odchyleń w danych, co pokazałem na bazie prostego zbioru jakim jest MNIST. Niestety przy przejściu do bardziej skomplikowanych próbek, model ten nie był w stanie znaleźć na tyle znaczących cech, które byłyby wystarczające do wykrywania zaburzeń z zadowalającą precyzją. Może być kilka przyczyn wystąpienia tego zjawiska, zaczynając od zbyt słabego modelu jakim jest sam autoenkoder po niewystarczającą obróbkę danych.

\section{Usprawnienia}

Skoro wyszło, że model zwraca przede wszystkim uwagę na jasność obrazków, to można byłoby zastosować normalizację w postaci wyrównywania histogramu (ang. histogram equalization). Metoda ta pozwala na zwiększenie kontrastu próbki, a dodatkowo rozciąga występowanie piksli na całą przestrzeń przez co zmienia się ich łączna suma wartości. Można wyobrazić sobie, że jasne obrazki zrobią się ciemniejsze i odwrotnie w przeciwnym przypadku. Jest szansa, że zmusiło by to model to położenia nacisku na inne cechy.

Innym pomysłem mogłoby być usprawnienie obróbki danych poprzez ograniczenie się jedynie do najbardziej znaczącej zawartości czaszki jakim jest mózg. W danych na jednych obrazkach znajdują się dodatkowo oczy, a na innych nie. Są to jasne obiekty, co mogło wpłynąć na reprezentację danych. Dodatkowo sama czaszka w formie kości też jest rzadka, więc usunięcie takich nieistotnych danych pozwoliłoby na sensowne ograniczenie danych.

Problemem może być sam rozmiar wycinka i brak jakichkolwiek dodatkowych informacji. W tym momencie model tylko na bazie samego obrazka musi go dobrze zrekonstruować, co wydaje się bardzo trudne szczególnie w tych partiach przy krawędziach, o których ma najmniej danych. Rozwiązaniem mogłoby być dorzucenie dodatkowych informacji do dekodera o sąsiedztwie takiego wycinka, co mogłoby pozytywnie wpłynąć na koszt rekonstrukcji.

