\chapter{Introduction}

 Czy wykorzystanie sztucznych sieci neuronowych może pomóc w diagnozowaniu nowotworów? W tej pracy będę chciał przedstawić podejście do tego problemu z wykorzystaniem wariacyjnego auto-enkodera. Dane na których będę bazował pochodzą z uniwersytetu Duke i przedstawiają zdjęcia rezonansowe głowy pacjentów z wykrytym nowotworem. I tu pojawia się jeden z pierwszych problemów, jakim jest niezbalansowanie danych. W ogólności liczba osób chorych jest zdecydowanie mniejsza od tych zdrowych, a dodatkowo samych komórek nowotworowych też jest mniej w porównaniu do pozostałych. W tym ma właśnie pomóc VAE, który stara się oszacować prawdopodobieństwo wystąpienia danej próbki, a skoro danych zmutowanych jest mniej, to ich prawdopodobieństwo również powinno być niższe. Będzie to jedna z podstaw użyta do klasyfikowania występowania raka.

%Can the use of artificial neural networks help diagnose cancer? In this work, I am going to present an approach to this problem using a Variational Autoencoder. The data on which I will be based come from the University of Duke and present resonant images of the head of patients with detected cancer. And here comes one of the first problems, which is the unbalancing of data. In general, the number of patients is much smaller than the healthy ones, and in addition, the cancer cells themselves are also less compared to the others. This is to be helped by VAE, which tries to estimate the probability of a given sample, and since the mutant data is less, their likelihood should also be lower.