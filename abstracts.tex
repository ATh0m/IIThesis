 \polishabstract {W tej pracy chciałem sprawdzić skuteczność wykorzystania metody uczenia bez nadzoru w problemie lokalizowania zmian nowotworowych na zdjęciach wykonanych metodą rezonansu magnetycznego. Bazowałem na podstawie obserwacji, że zmiany patologiczne są rzadkie i są pewnym odstępstwem od normy. Zdecydowałem się na użycie autoenkodera wariacyjnego, który pozwala oszacować prawdopodobieństwo wystąpienia danej próbki, co pozwoliłoby mi na ich klasyfikowanie. Pomysł przetestowałem na danych syntetycznych, wykorzystując do tego zbiór MNIST. Otrzymane wyniki okazały się być zadowalające i zachęcały do przeprowadzenia dalszych eksperymentów już na danych medycznych. Niestety w tym przypadku nie można było uznać tego za sukces, a według mojej analizy model nauczył się jedynie zwracać uwagę na prostą własność, jaką jest jasność próbki. Na tym zakończyłem moją pracę i przygotowałem propozycje rozwiązań, które mogłyby według mnie pozytywnie wpłynąć na poprawę rezultatów. }
