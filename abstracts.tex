 \polishabstract { 
Problemem, który zainspirował mnie do napisania tej pracy było lokalizowanie zmian nowotworowych w zdjęciach wykonanych metodą rezonansu magnetycznego. O ile można podejść do tego zadania od strony nadzorowanej, czyli bazowania na danych przeanalizowanych wcześniej przez specjalistów, gdzie każdy przypadek musiał zostać ręcznie obejrzany i oznaczony, to mimo wielu jego zalet jak chociażby korzystanie z rzeczywistej wiedzy eksperckiej, zbiór taki jest bardzo kosztowny w przygotowaniu i dalszym rozwoju, a dodatkowo jest on ograniczony do przykładowo pojedynczej partii ciała. Zauważając te wady oraz łącząc je z obserwacją, że zmiany patologiczne tak na prawdę są rzadkie i są pewnym odstępstwem od normy, chciałem spróbować skorzystać z metody nienadzorowanej, w które to model nauczyłby się oszacowywać prawdopodobieństwo występowania pojedynczej próbki w pewnym kontekście. Przy takim podejściu mógłbym oznaczać obserwacje mało prawdopodobne jako właśnie te nowotworowe. Model, który zdecydowałem się wykorzystać do tego zadania to autoenkoder wariacyjny (ang. Variational Autoencoder), łączący sztuczne sieci neuronowe z modelowaniem probabilistycznym. Pomysł przetestowałem początkowo na danych syntetycznych, wykorzystując do tego zbiór MNIST. Otrzymane wyniki okazały się być zadowalające i zachęcały do przeprowadzenia dalszych eksperymentów już na danych medycznych. Niestety w tym przypadku nie można było uznać tego za sukces, a według mojej analizy model nauczył się jedynie zwracać uwagę na prostą własność, jaką jest jasność próbki. Na tym zakończyłem moją pracę i przygotowałem propozycję rozwiązań, który mogłyby według mnie pozytywnie wpłynąć na poprawę rezultatów.
}
